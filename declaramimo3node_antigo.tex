%%%%%%%%%%%%%%%%%%%%%%%%%%%%%%%%%%%%%%%%%%%%%%%%%%%%%%%%%%%%%%%%%%%%%%%%%%%
%%%%%%%%%%%%%%%%%%%%%%%%%%%%%%%%%%%%%%%%%%%%%%%%%%%%%%%%%%%%%%%%%%%%%%%%%%%
%%%%%%%%%%%%%%%%%%%%%%%%%%%%%%%%%%%%%%%%%%%%%%%%%%%%%%%%%%%%%%%%%%%%%%%%%%%
% Modified from the example in
% http://www.texample.net/tikz/examples/d-flip-flops-and-shift-register/
\makeatletter

\pgfdeclareshape{mimotresdireita}{
  % The 'minimum width' and 'minimum height' keys, not the content, determine
  % the size


  % \pgf@y=.\ht\pgfnodeparttextbox % height of the box
  % \pgf@x=.\wd\pgfnodeparttextbox % width of the box
  % \setlength{\pgf@xa}{\pgfshapeminwidth}
  % \ifdim\pgf@x<.5\pgf@xa
  % \pgf@x=.5\pgf@xa
  % \fi

  \savedanchor\northeast{%
    \pgf@y=.5\ht\pgfnodeparttextbox % height of the box, ignoring the depth
    \pgf@x=.5\wd\pgfnodeparttextbox % width of the box
    \setlength{\pgf@xa}{\pgfshapeminwidth}
    \ifdim\pgf@x<.5\pgf@xa
    \pgf@x=.5\pgf@xa
    \fi
    \setlength{\pgf@ya}{\pgfshapeminheight}
    \ifdim\pgf@y<.5\pgf@ya
    \pgf@y=.5\pgf@ya
    \fi
  }
  % This is redundant, but makes some things easier:
  \savedanchor\southwest{%
    \pgf@y=-.5\ht\pgfnodeparttextbox % height of the box, ignoring the depth
    \pgf@x=-.5\wd\pgfnodeparttextbox % width of the box
    \setlength{\pgf@xa}{-\pgfshapeminwidth}
    \ifdim\pgf@x>.5\pgf@xa
    \pgf@x=.5\pgf@xa
    \fi
    \setlength{\pgf@ya}{-\pgfshapeminheight}
    \ifdim\pgf@y>.5\pgf@ya
    \pgf@y=.5\pgf@ya
    \fi
  }

  \savedanchor\firstantennabase{%
    \pgf@y=.5\ht\pgfnodeparttextbox % height of the box, ignoring the depth
    \pgf@x=.5\wd\pgfnodeparttextbox % width of the box
    \setlength{\pgf@xa}{\pgfshapeminwidth}
    \ifdim\pgf@x<.5\pgf@xa
    \pgf@x=.5\pgf@xa
    \fi
    \setlength{\pgf@ya}{\pgfshapeminheight}
    \ifdim\pgf@y<.3\pgf@ya
    \pgf@y=.3\pgf@ya
    \fi
  }

  \savedanchor\secondantennabase{%
    \pgf@y=.5\ht\pgfnodeparttextbox % height of the box, ignoring the depth
    \pgf@x=.5\wd\pgfnodeparttextbox % width of the box
    \setlength{\pgf@xa}{\pgfshapeminwidth}
    \ifdim\pgf@x<.5\pgf@xa
    \pgf@x=.5\pgf@xa
    \fi
    \pgf@y=0.0\pgf@y
  }
  
  \savedanchor\thirdantennabase{%
    \pgf@y=.5\ht\pgfnodeparttextbox % height of the box, ignoring the depth
    \pgf@x=.5\wd\pgfnodeparttextbox % width of the box
    \setlength{\pgf@xa}{\pgfshapeminwidth}
    \ifdim\pgf@x<.5\pgf@xa
    \pgf@x=.5\pgf@xa
    \fi
    \setlength{\pgf@ya}{\pgfshapeminheight}
    \ifdim\pgf@y>-.3\pgf@ya
    \pgf@y=-.3\pgf@ya
    \fi
  }

  % Inherit from rectangle
  \inheritanchorborder[from=rectangle]

  % Define same anchor a normal rectangle has
  \anchor{center}{\pgfpointorigin}
  \anchor{north}{\northeast \pgf@x=0pt}
  \anchor{east}{\northeast \pgf@y=0pt}
  \anchor{south}{\southwest \pgf@x=0pt}
  \anchor{west}{\southwest \pgf@y=0pt}
  \anchor{north east}{\northeast}
  \anchor{north west}{\northeast \pgf@x=-\pgf@x}
  \anchor{south west}{\southwest}
  \anchor{south east}{\southwest \pgf@x=-\pgf@x}
  \anchor{text}{
    \pgfpointorigin
    \advance\pgf@x by -.5\wd\pgfnodeparttextbox%
    \advance\pgf@y by -.5\ht\pgfnodeparttextbox%
    \advance\pgf@y by +.5\dp\pgfnodeparttextbox%
  }

  % \anchor{first antenna base}{
  %   \pgf@process{\northeast}%
  %   \pgf@y=0.6\pgf@y%
  % }
  \anchor{first antenna base start}{\firstantennabase}
  \anchor{second antenna base start}{\secondantennabase}
  \anchor{third antenna base start}{\thirdantennabase}

  \anchor{first antenna base end}{
    \firstantennabase
    \advance\pgf@x by 0.3cm%
    \advance\pgf@y by 0.2cm%
  }
  \anchor{second antenna base end}{
    \secondantennabase
    \advance\pgf@x by 0.3cm%
    \advance\pgf@y by 0.2cm%
  }
  \anchor{third antenna base end}{
    \thirdantennabase
    \advance\pgf@x by 0.3cm%
    \advance\pgf@y by 0.2cm%
  }


  \anchor{first antenna}{
    \firstantennabase
    \advance\pgf@x by 0.3cm%
    \advance\pgf@y by 0.4cm%
  }
  \anchor{second antenna}{
    \secondantennabase
    \advance\pgf@x by 0.3cm%
    \advance\pgf@y by 0.4cm%
  }
  \anchor{third antenna}{
    \thirdantennabase
    \advance\pgf@x by 0.3cm%
    \advance\pgf@y by 0.4cm%
  }

  % Define anchors for signal ports

  \anchor{A}{
    \pgf@process{\firstantennabase}%
    % \pgf@x=0\pgf@x%
    % \pgf@y=0.6\pgf@y%
  }

  \anchor{B}{
    \pgf@process{\secondantennabase}%
    % \pgf@x=0\pgf@x%
    % \pgf@y=0\pgf@y%
  }

  \anchor{C}{
    \pgf@process{\thirdantennabase}%
    % \pgf@x=0\pgf@x%
    % \pgf@y=-0.6\pgf@y%
  }


  % Draw the rectangle box and the three circles
  \backgroundpath{
    % Rectangle box
    \pgfpathrectanglecorners{\southwest}{\northeast}
    
    %%%%%%% Draw the first antenna %%%%%%%%%%%%%%%%%%%%%%%%%%%%%%%%%%%%%%%%
    %%%%\pgf@anchor@mimotresdireita@A
    % Você pode usar \pgf@anchor@SHAPE_NAME@ANCHOR_NAME para mudar o \pgf@x
    % e o \pgf@y para as coordenadas de uma âncora. Só que se você resolver
    % mudar o nome do shape depois vai ter que mudar todas as linhas que
    % contem o SHAPE_NAME. Melhor usar o csname como na linha abaixo.
    \csname pgf@anchor@\shape@name @A\endcsname
    \pgf@xa=\pgf@x \pgf@ya=\pgf@y
    \pgfpathmoveto{\pgfpoint{\pgf@x}{\pgf@y}}

    \advance\pgf@xa by 0.3cm
    \pgfpathlineto{\pgfpoint{\pgf@xa}{\pgf@ya}}
    
    \pgf@xb=\pgf@xa \pgf@yb=\pgf@ya
    \advance\pgf@yb by 0.2cm
    \pgfpathlineto{\pgfpoint{\pgf@xb}{\pgf@yb}}
    
    \pgf@xc=\pgf@xb \pgf@yc=\pgf@yb
    \advance\pgf@xc by 0.2cm
    \advance\pgf@yc by 0.3cm
    \pgfpathlineto{\pgfpoint{\pgf@xc}{\pgf@yc}}

    \pgf@xa=\pgf@xc \pgf@ya=\pgf@yc
    \advance\pgf@xa by -0.4cm
    \pgfpathlineto{\pgfpoint{\pgf@xa}{\pgf@ya}}
    
    \pgf@xb=\pgf@xa \pgf@yb=\pgf@ya
    \advance\pgf@xb by 0.2cm
    \advance\pgf@yb by -0.3cm
    \pgfpathlineto{\pgfpoint{\pgf@xb}{\pgf@yb}}
    %%%%%%%%%%%%%%%%%%%%%%%%%%%%%%%%%%%%%%%%%%%%%%%%%%%%%%%%%%%%%%%%%%%%%%%


    %%%%% Draw the second antenna %%%%%%%%%%%%%%%%%%%%%%%%%%%%%%%%%%%%%%%%%
    %\pgf@anchor@mimotresdireita@B
    \csname pgf@anchor@\shape@name @B\endcsname
    \pgf@xa=\pgf@x \pgf@ya=\pgf@y
    \pgfpathmoveto{\pgfpoint{\pgf@x}{\pgf@y}}

    \advance\pgf@xa by 0.3cm
    \pgfpathlineto{\pgfpoint{\pgf@xa}{\pgf@ya}}
    
    \pgf@xb=\pgf@xa \pgf@yb=\pgf@ya
    \advance\pgf@yb by 0.2cm
    \pgfpathlineto{\pgfpoint{\pgf@xb}{\pgf@yb}}
    
    \pgf@xc=\pgf@xb \pgf@yc=\pgf@yb
    \advance\pgf@xc by 0.2cm
    \advance\pgf@yc by 0.3cm
    \pgfpathlineto{\pgfpoint{\pgf@xc}{\pgf@yc}}

    \pgf@xa=\pgf@xc \pgf@ya=\pgf@yc
    \advance\pgf@xa by -0.4cm
    \pgfpathlineto{\pgfpoint{\pgf@xa}{\pgf@ya}}
    
    \pgf@xb=\pgf@xa \pgf@yb=\pgf@ya
    \advance\pgf@xb by 0.2cm
    \advance\pgf@yb by -0.3cm
    \pgfpathlineto{\pgfpoint{\pgf@xb}{\pgf@yb}}
    %%%%%%%%%%%%%%%%%%%%%%%%%%%%%%%%%%%%%%%%%%%%%%%%%%%%%%%%%%%%%%%%%%%%%%%
    

    %%%%% Draw the third antenna %%%%%%%%%%%%%%%%%%%%%%%%%%%%%%%%%%%%%%%%%%
    %\pgf@anchor@mimotresdireita@C
    \csname pgf@anchor@\shape@name @C\endcsname
    \pgf@xa=\pgf@x \pgf@ya=\pgf@y
    \pgfpathmoveto{\pgfpoint{\pgf@x}{\pgf@y}}

    \advance\pgf@xa by 0.3cm
    \pgfpathlineto{\pgfpoint{\pgf@xa}{\pgf@ya}}
    
    \pgf@xb=\pgf@xa \pgf@yb=\pgf@ya
    \advance\pgf@yb by 0.2cm
    \pgfpathlineto{\pgfpoint{\pgf@xb}{\pgf@yb}}
    
    \pgf@xc=\pgf@xb \pgf@yc=\pgf@yb
    \advance\pgf@xc by 0.2cm
    \advance\pgf@yc by 0.3cm
    \pgfpathlineto{\pgfpoint{\pgf@xc}{\pgf@yc}}

    \pgf@xa=\pgf@xc \pgf@ya=\pgf@yc
    \advance\pgf@xa by -0.4cm
    \pgfpathlineto{\pgfpoint{\pgf@xa}{\pgf@ya}}
    
    \pgf@xb=\pgf@xa \pgf@yb=\pgf@ya
    \advance\pgf@xb by 0.2cm
    \advance\pgf@yb by -0.3cm
    \pgfpathlineto{\pgfpoint{\pgf@xb}{\pgf@yb}}
    %%%%%%%%%%%%%%%%%%%%%%%%%%%%%%%%%%%%%%%%%%%%%%%%%%%%%%%%%%%%%%%%%%%%%%%
  }
}


% Key to add font macros to the current font
\tikzset{add font/.code={\expandafter\def\expandafter\tikz@textfont\expandafter{\tikz@textfont#1}}} 

% Define default style for this node
%\tikzset{flip flop/port labels/.style={font=\sffamily\scriptsize}}
\tikzset{every mimotresdireita node/.style={draw,minimum width=1.5cm,minimum 
height=2.12132034375cm,thick,inner sep=1mm,outer sep=0pt,cap=round,add 
font=\sffamily}}

\makeatother
%%%%%%%%%%%%%%%%%%%%%%%%%%%%%%%%%%%%%%%%%%%%%%%%%%%%%%%%%%%%%%%%%%%%%%%%%%%
%%%%%%%%%%%%%%%%%%%%%%%%%%%%%%%%%%%%%%%%%%%%%%%%%%%%%%%%%%%%%%%%%%%%%%%%%%%
%%%%%%%%%%%%%%%%%%%%%%%%%%%%%%%%%%%%%%%%%%%%%%%%%%%%%%%%%%%%%%%%%%%%%%%%%%%




%%%%%%%%%%%%%%%%%%%%%%%%%%%%%%%%%%%%%%%%%%%%%%%%%%%%%%%%%%%%%%%%%%%%%%%%%%%
%%%%%%%%%%%%%%%%%%%%%%%%%%%%%%%%%%%%%%%%%%%%%%%%%%%%%%%%%%%%%%%%%%%%%%%%%%%
%%%%%%%%%%%%%%%%%%%%%%%%%%%%%%%%%%%%%%%%%%%%%%%%%%%%%%%%%%%%%%%%%%%%%%%%%%%
% Modified from the example in
% http://www.texample.net/tikz/examples/d-flip-flops-and-shift-register/
\makeatletter

\pgfdeclareshape{mimotresesquerda}{
  % The 'minimum width' and 'minimum height' keys, not the content, determine
  % the size


  % \pgf@y=.\ht\pgfnodeparttextbox % height of the box
  % \pgf@x=.\wd\pgfnodeparttextbox % width of the box
  % \setlength{\pgf@xa}{\pgfshapeminwidth}
  % \ifdim\pgf@x<.5\pgf@xa
  % \pgf@x=.5\pgf@xa
  % \fi

  \savedanchor\northeast{%
    \pgf@y=.5\ht\pgfnodeparttextbox % height of the box, ignoring the depth
    \pgf@x=.5\wd\pgfnodeparttextbox % width of the box
    \setlength{\pgf@xa}{\pgfshapeminwidth}
    \ifdim\pgf@x<.5\pgf@xa
    \pgf@x=.5\pgf@xa
    \fi
    \setlength{\pgf@ya}{\pgfshapeminheight}
    \ifdim\pgf@y<.5\pgf@ya
    \pgf@y=.5\pgf@ya
    \fi
  }
  % This is redundant, but makes some things easier:
  \savedanchor\southwest{%
    \pgf@y=-.5\ht\pgfnodeparttextbox % height of the box, ignoring the depth
    \pgf@x=-.5\wd\pgfnodeparttextbox % width of the box
    \setlength{\pgf@xa}{-\pgfshapeminwidth}
    \ifdim\pgf@x>.5\pgf@xa
    \pgf@x=.5\pgf@xa
    \fi
    \setlength{\pgf@ya}{-\pgfshapeminheight}
    \ifdim\pgf@y>.5\pgf@ya
    \pgf@y=.5\pgf@ya
    \fi
  }

  \savedanchor\firstantennabase{%
    \pgf@y=.5\ht\pgfnodeparttextbox % height of the box, ignoring the depth
    \pgf@x=-.5\wd\pgfnodeparttextbox % width of the box
    \setlength{\pgf@xa}{\pgfshapeminwidth}
    \ifdim\pgf@x>-.5\pgf@xa
    \pgf@x=-.5\pgf@xa
    \fi
    \setlength{\pgf@ya}{\pgfshapeminheight}
    \ifdim\pgf@y<.3\pgf@ya
    \pgf@y=.3\pgf@ya
    \fi
  }

  \savedanchor\secondantennabase{%
    \pgf@y=.5\ht\pgfnodeparttextbox % height of the box, ignoring the depth
    \pgf@x=-.5\wd\pgfnodeparttextbox % width of the box
    \setlength{\pgf@xa}{\pgfshapeminwidth}
    \ifdim\pgf@x>-.5\pgf@xa
    \pgf@x=-.5\pgf@xa
    \fi
    \pgf@y=0.0\pgf@y
  }
  
  \savedanchor\thirdantennabase{%
    \pgf@y=.5\ht\pgfnodeparttextbox % height of the box, ignoring the depth
    \pgf@x=-.5\wd\pgfnodeparttextbox % width of the box
    \setlength{\pgf@xa}{\pgfshapeminwidth}
    \ifdim\pgf@x>-.5\pgf@xa
    \pgf@x=-.5\pgf@xa
    \fi
    \setlength{\pgf@ya}{\pgfshapeminheight}
    \ifdim\pgf@y>-.3\pgf@ya
    \pgf@y=-.3\pgf@ya
    \fi
  }

  % Inherit from rectangle
  \inheritanchorborder[from=rectangle]

  % Define same anchor a normal rectangle has
  \anchor{center}{\pgfpointorigin}
  \anchor{north}{\northeast \pgf@x=0pt}
  \anchor{east}{\northeast \pgf@y=0pt}
  \anchor{south}{\southwest \pgf@x=0pt}
  \anchor{west}{\southwest \pgf@y=0pt}
  \anchor{north east}{\northeast}
  \anchor{north west}{\northeast \pgf@x=-\pgf@x}
  \anchor{south west}{\southwest}
  \anchor{south east}{\southwest \pgf@x=-\pgf@x}
  \anchor{text}{
    \pgfpointorigin
    \advance\pgf@x by -.5\wd\pgfnodeparttextbox%
    \advance\pgf@y by -.5\ht\pgfnodeparttextbox%
    \advance\pgf@y by +.5\dp\pgfnodeparttextbox%
  }

  % \anchor{first antenna base}{
  %   \pgf@process{\northeast}%
  %   \pgf@y=0.6\pgf@y%
  % }
  \anchor{first antenna base start}{\firstantennabase}
  \anchor{second antenna base start}{\secondantennabase}
  \anchor{third antenna base start}{\thirdantennabase}

  \anchor{first antenna base end}{
    \firstantennabase
    \advance\pgf@x by -0.3cm%
    \advance\pgf@y by 0.2cm%
  }
  \anchor{second antenna base end}{
    \secondantennabase
    \advance\pgf@x by -0.3cm%
    \advance\pgf@y by 0.2cm%
  }
  \anchor{third antenna base end}{
    \thirdantennabase
    \advance\pgf@x by -0.3cm%
    \advance\pgf@y by 0.2cm%
  }


  \anchor{first antenna}{
    \firstantennabase
    \advance\pgf@x by -0.3cm%
    \advance\pgf@y by 0.4cm%
  }
  \anchor{second antenna}{
    \secondantennabase
    \advance\pgf@x by -0.3cm%
    \advance\pgf@y by 0.4cm%
  }
  \anchor{third antenna}{
    \thirdantennabase
    \advance\pgf@x by -0.3cm%
    \advance\pgf@y by 0.4cm%
  }


  % Define helper anchors
  \anchor{A}{
    \pgf@process{\firstantennabase}%
    % \pgf@x=0\pgf@x%
    % \pgf@y=0.6\pgf@y%
  }

  \anchor{B}{
    \pgf@process{\secondantennabase}%
    % \pgf@x=0\pgf@x%
    % \pgf@y=0\pgf@y%
  }

  \anchor{C}{
    \pgf@process{\thirdantennabase}%
    % \pgf@x=0\pgf@x%
    % \pgf@y=-0.6\pgf@y%
  }


  % Draw the rectangle box and the three circles
  \backgroundpath{
    % Rectangle box
    \pgfpathrectanglecorners{\southwest}{\northeast}
        %%%%%%% Draw the first antenna %%%%%%%%%%%%%%%%%%%%%%%%%%%%%%%%%%%%%%%%
    % \pgf@anchor@mimotresesquerda@A
    \csname pgf@anchor@\shape@name @A\endcsname
    \pgf@xa=\pgf@x \pgf@ya=\pgf@y
    \pgfpathmoveto{\pgfpoint{\pgf@x}{\pgf@y}}

    \advance\pgf@xa by -0.3cm
    \pgfpathlineto{\pgfpoint{\pgf@xa}{\pgf@ya}}
    
    \pgf@xb=\pgf@xa \pgf@yb=\pgf@ya
    \advance\pgf@yb by 0.2cm
    \pgfpathlineto{\pgfpoint{\pgf@xb}{\pgf@yb}}
    
    \pgf@xc=\pgf@xb \pgf@yc=\pgf@yb
    \advance\pgf@xc by 0.2cm
    \advance\pgf@yc by 0.3cm
    \pgfpathlineto{\pgfpoint{\pgf@xc}{\pgf@yc}}

    \pgf@xa=\pgf@xc \pgf@ya=\pgf@yc
    \advance\pgf@xa by -0.4cm
    \pgfpathlineto{\pgfpoint{\pgf@xa}{\pgf@ya}}
    
    \pgf@xb=\pgf@xa \pgf@yb=\pgf@ya
    \advance\pgf@xb by 0.2cm
    \advance\pgf@yb by -0.3cm
    \pgfpathlineto{\pgfpoint{\pgf@xb}{\pgf@yb}}
    %%%%%%%%%%%%%%%%%%%%%%%%%%%%%%%%%%%%%%%%%%%%%%%%%%%%%%%%%%%%%%%%%%%%%%%


    %%%%% Draw the second antenna %%%%%%%%%%%%%%%%%%%%%%%%%%%%%%%%%%%%%%%%%
    % \pgf@anchor@mimotresesquerda@B
    \csname pgf@anchor@\shape@name @B\endcsname
    \pgf@xa=\pgf@x \pgf@ya=\pgf@y
    \pgfpathmoveto{\pgfpoint{\pgf@x}{\pgf@y}}

    \advance\pgf@xa by -0.3cm
    \pgfpathlineto{\pgfpoint{\pgf@xa}{\pgf@ya}}
    
    \pgf@xb=\pgf@xa \pgf@yb=\pgf@ya
    \advance\pgf@yb by 0.2cm
    \pgfpathlineto{\pgfpoint{\pgf@xb}{\pgf@yb}}
    
    \pgf@xc=\pgf@xb \pgf@yc=\pgf@yb
    \advance\pgf@xc by 0.2cm
    \advance\pgf@yc by 0.3cm
    \pgfpathlineto{\pgfpoint{\pgf@xc}{\pgf@yc}}

    \pgf@xa=\pgf@xc \pgf@ya=\pgf@yc
    \advance\pgf@xa by -0.4cm
    \pgfpathlineto{\pgfpoint{\pgf@xa}{\pgf@ya}}
    
    \pgf@xb=\pgf@xa \pgf@yb=\pgf@ya
    \advance\pgf@xb by 0.2cm
    \advance\pgf@yb by -0.3cm
    \pgfpathlineto{\pgfpoint{\pgf@xb}{\pgf@yb}}
    %%%%%%%%%%%%%%%%%%%%%%%%%%%%%%%%%%%%%%%%%%%%%%%%%%%%%%%%%%%%%%%%%%%%%%%
    

    %%%%% Draw the third antenna %%%%%%%%%%%%%%%%%%%%%%%%%%%%%%%%%%%%%%%%%%
    % \pgf@anchor@mimotresesquerda@C
    \csname pgf@anchor@\shape@name @C\endcsname
    \pgf@xa=\pgf@x \pgf@ya=\pgf@y
    \pgfpathmoveto{\pgfpoint{\pgf@x}{\pgf@y}}

    \advance\pgf@xa by -0.3cm
    \pgfpathlineto{\pgfpoint{\pgf@xa}{\pgf@ya}}
    
    \pgf@xb=\pgf@xa \pgf@yb=\pgf@ya
    \advance\pgf@yb by 0.2cm
    \pgfpathlineto{\pgfpoint{\pgf@xb}{\pgf@yb}}
    
    \pgf@xc=\pgf@xb \pgf@yc=\pgf@yb
    \advance\pgf@xc by 0.2cm
    \advance\pgf@yc by 0.3cm
    \pgfpathlineto{\pgfpoint{\pgf@xc}{\pgf@yc}}

    \pgf@xa=\pgf@xc \pgf@ya=\pgf@yc
    \advance\pgf@xa by -0.4cm
    \pgfpathlineto{\pgfpoint{\pgf@xa}{\pgf@ya}}
    
    \pgf@xb=\pgf@xa \pgf@yb=\pgf@ya
    \advance\pgf@xb by 0.2cm
    \advance\pgf@yb by -0.3cm
    \pgfpathlineto{\pgfpoint{\pgf@xb}{\pgf@yb}}
    %%%%%%%%%%%%%%%%%%%%%%%%%%%%%%%%%%%%%%%%%%%%%%%%%%%%%%%%%%%%%%%%%%%%%%%

  }
}

% Key to add font macros to the current font
% \tikzset{add font/.code={\expandafter\def\expandafter\tikz@textfont\expandafter{\tikz@textfont#1}}} 

% Define default style for this node
%\tikzset{flip flop/port labels/.style={font=\sffamily\scriptsize}}
\tikzset{every mimotresesquerda node/.style={draw,minimum width=1.5cm,minimum 
height=2.12132034375cm,thick,inner sep=1mm,outer sep=0pt,cap=round,add 
font=\sffamily}}

\makeatother
%%%%%%%%%%%%%%%%%%%%%%%%%%%%%%%%%%%%%%%%%%%%%%%%%%%%%%%%%%%%%%%%%%%%%%%%%%%
%%%%%%%%%%%%%%%%%%%%%%%%%%%%%%%%%%%%%%%%%%%%%%%%%%%%%%%%%%%%%%%%%%%%%%%%%%%
%%%%%%%%%%%%%%%%%%%%%%%%%%%%%%%%%%%%%%%%%%%%%%%%%%%%%%%%%%%%%%%%%%%%%%%%%%%

%%% Local Variables: 
%%% mode: latex
%%% TeX-master: "figs/testnewshape.tex"
%%% TeX-PDF-mode: t
%%% End: 
