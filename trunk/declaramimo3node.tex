%%%%%%%%%%%%%%%%%%%%%%%%%%%%%%%%%%%%%%%%%%%%%%%%%%%%%%%%%%%%%%%%%%%%%%%%%%%
%%%%%%%%%%%%%%%%%%%%%%%%%%%%%%%%%%%%%%%%%%%%%%%%%%%%%%%%%%%%%%%%%%%%%%%%%%%
%%%%%%%%%%%%%%%%%%%%%%%%%%%%%%%%%%%%%%%%%%%%%%%%%%%%%%%%%%%%%%%%%%%%%%%%%%%
% \pgfmathparse{1+1}\pgfmathresult
%keys no manual do pgf -> pag 481


% Modified from the example in
% http://www.texample.net/tikz/examples/d-flip-flops-and-shift-register/
\makeatletter


\pgfset{%/pgf/.cd,%
  antenna offset/.initial=0.3,%
% antennaoffset/.default=0.3,%
%  antennaoffset/.code=#1,
  espelho/.initial=1
  espelho/.code={\pgfkeyssetvalue{/pgf/espelho}{-\pgfkeysvalueof{/pgf/espelho}}}
  %espelho/.code={\pgfkeyssetvalue{/pgf/antennaoffset}{-0.3cm}}
}


% Macro com código comum para o desenho das antenas. Antes de chamar esse
% macro, sete os valores de pgf@x e pgf@y apropriadamente. O argumento é o
% tamanho da reta que vai do ponto inicial até o ponto em que a antenna
% sobe.
\def\desenhaantenna{
  % Guardo as posições atuais, que devem ser o ponto onde a antena começa a
  % ser desenhada
  \pgf@xa=\pgf@x \pgf@ya=\pgf@y
  \pgfpathmoveto{\pgfpoint{\pgf@x}{\pgf@y}}

  % #1 deve ser positivo para antena na direita e negativo caso contrário
  \advance\pgf@xa by \antennaoffset cm
  \pgfpathlineto{\pgfpoint{\pgf@xa}{\pgf@ya}}
  \pgfpathclose
  \pgfmoveto{\pgfpoint{\pgf@xa}{\pgf@ya}}

  \pgf@xb=\pgf@xa \pgf@yb=\pgf@ya
  \advance\pgf@yb by 0.2cm
  \pgfpathlineto{\pgfpoint{\pgf@xb}{\pgf@yb}}
  
  \pgf@xc=\pgf@xb \pgf@yc=\pgf@yb
  \advance\pgf@xc by 0.2cm
  \advance\pgf@yc by 0.3cm
  \pgfpathlineto{\pgfpoint{\pgf@xc}{\pgf@yc}}

  \pgf@xa=\pgf@xc \pgf@ya=\pgf@yc
  \advance\pgf@xa by -0.4cm
  \pgfpathlineto{\pgfpoint{\pgf@xa}{\pgf@ya}}
  
  \pgf@xb=\pgf@xa \pgf@yb=\pgf@ya
  \advance\pgf@xb by 0.2cm
  \advance\pgf@yb by -0.3cm
  \pgfpathlineto{\pgfpoint{\pgf@xb}{\pgf@yb}}
}


\pgfdeclareshape{mimotresdireita}{
  \savedmacro\antennaoffset{\def\antennaoffset{\pgfkeysvalueof{/pgf/antenna offset}}}
  \savedmacro\espelho{\def\espelho{\pgfkeysvalueof{/pgf/espelho}}}

  % The 'minimum width' and 'minimum height' keys, not the content, determine
  % the size

  % Inherit saved anchor from rectangle (\northeast and \southwest)
  \inheritsavedanchors[from=rectangle]

  % Inherit anchor border from rectangle
  \inheritanchorborder[from=rectangle]

  % Inherit normal anchors from rectangle
  \inheritanchor[from=rectangle]{center}
  \inheritanchor[from=rectangle]{north}
  \inheritanchor[from=rectangle]{east}
  \inheritanchor[from=rectangle]{south}
  \inheritanchor[from=rectangle]{west}
  \inheritanchor[from=rectangle]{north east}
  \inheritanchor[from=rectangle]{north west}
  \inheritanchor[from=rectangle]{south west}
  \inheritanchor[from=rectangle]{south east}
  \inheritanchor[from=rectangle]{text}
  \inheritanchor[from=rectangle]{base}
  \inheritanchor[from=rectangle]{mid}
  % Posso herdar outras como "mid west", "base west", etc. Veja no código do rectangle

  % Define anchors for the start of the antennas
  \anchor{A}{
    \pgf@process{\northeast}
    \pgf@y=.5\pgf@y
  }

  \anchor{C}{
    \pgf@process{\southwest}
    \pgf@ya=.9\pgf@y % Na coordenada southwest o valor de y é negativo
    \pgf@process{\northeast}
    \pgf@y=\pgf@ya
  }

  % B is in the middle of A and C
  \anchor{B}{
    \csname pgf@anchor@mimotresdireita@C\endcsname
    \pgf@ya=.5\pgf@y
    \csname pgf@anchor@mimotresdireita@A\endcsname
    \pgf@y=.5\pgf@y
    \advance\pgf@y by \pgf@ya
  }

  % Define some more useful anchors
  \anchor{first antenna base start}{\pgf@anchor@mimotresdireita@A}
  \anchor{second antenna base start}{\pgf@anchor@mimotresdireita@B}
  \anchor{third antenna base start}{\pgf@anchor@mimotresdireita@C}

  \anchor{first antenna base end}{
    % Preciso usar o \csname ... \endcsname porque a âncora "first antenna
    % base start" possui espaços. Note que não pode ter espaço entre o
    % final de "first antenna base start" e o \endcsname
    \csname pgf@anchor@mimotresdireita@first antenna base start\endcsname
    \advance\pgf@x by \antennaoffset cm%
    \advance\pgf@y by 0.2cm%
  }
  \anchor{second antenna base end}{
    \csname pgf@anchor@mimotresdireita@second antenna base start\endcsname
    \advance\pgf@x by \antennaoffset cm%
    \advance\pgf@y by 0.2cm%
  }
  \anchor{third antenna base end}{
    \csname pgf@anchor@mimotresdireita@third antenna base start\endcsname
    \advance\pgf@x by \antennaoffset cm%
    \advance\pgf@y by 0.2cm%
  }
  \anchor{first antenna}{
    \csname pgf@anchor@mimotresdireita@first antenna base start\endcsname
    \advance\pgf@x by \antennaoffset cm%
    \advance\pgf@y by 0.4cm%
  }
  \anchor{second antenna}{
    \csname pgf@anchor@mimotresdireita@second antenna base start\endcsname
    \advance\pgf@x by \antennaoffset cm%
    \advance\pgf@y by 0.4cm%
  }
  \anchor{third antenna}{
    \csname pgf@anchor@mimotresdireita@third antenna base start\endcsname
    \advance\pgf@x by \antennaoffset cm%
    \advance\pgf@y by 0.4cm%
  }


  % Draw the rectangle box and the antennas
  \backgroundpath{
    % Rectangle box
    \pgfpathrectanglecorners{\southwest}{\northeast}
    %%%%%%% Draw the first antenna %%%%%%%%%%%%%%%%%%%%%%%%%%%%%%%%%%%%%%%%
    % Você pode usar \pgf@anchor@SHAPE_NAME@ANCHOR_NAME para mudar o \pgf@x
    % e o \pgf@y para as coordenadas de uma âncora. Só que se você resolver
    % mudar o nome do shape depois vai ter que mudar todas as linhas que
    % contem o SHAPE_NAME. Melhor usar o csname como na linha abaixo.
    \csname pgf@anchor@\shape@name @A\endcsname
    %\pgf@anchor@mimotresdireita@A
    \desenhaantenna
    %%%%%%%%%%%%%%%%%%%%%%%%%%%%%%%%%%%%%%%%%%%%%%%%%%%%%%%%%%%%%%%%%%%%%%%


    %%%%% Draw the second antenna %%%%%%%%%%%%%%%%%%%%%%%%%%%%%%%%%%%%%%%%%
    %% \pgf@anchor@mimotresdireita@B
    \csname pgf@anchor@\shape@name @B\endcsname
    \desenhaantenna
    %%%%%%%%%%%%%%%%%%%%%%%%%%%%%%%%%%%%%%%%%%%%%%%%%%%%%%%%%%%%%%%%%%%%%%%
    

    %%%%% Draw the third antenna %%%%%%%%%%%%%%%%%%%%%%%%%%%%%%%%%%%%%%%%%%
    %% \pgf@anchor@mimotresdireita@C
    \csname pgf@anchor@\shape@name @C\endcsname
    \desenhaantenna
    %%%%%%%%%%%%%%%%%%%%%%%%%%%%%%%%%%%%%%%%%%%%%%%%%%%%%%%%%%%%%%%%%%%%%%%
  }
}


% Key to add font macros to the current font
\tikzset{add font/.code={\expandafter\def\expandafter\tikz@textfont\expandafter{\tikz@textfont#1}}} 

% Define default style for this node
%\tikzset{flip flop/port labels/.style={font=\sffamily\scriptsize}}
\tikzset{every mimotresdireita node/.style={draw,minimum width=1.5cm,minimum 
height=2cm,thick,inner sep=1mm,outer sep=0pt,cap=round,add 
font=\sffamily}}
% \tikzset{every mimotresdireita node/.style={draw,minimum width=1.5cm,minimum 
% height=2.12132034375cm,thick,inner sep=1mm,outer sep=0pt,cap=round,add 
% font=\sffamily}}

\makeatother
%%%%%%%%%%%%%%%%%%%%%%%%%%%%%%%%%%%%%%%%%%%%%%%%%%%%%%%%%%%%%%%%%%%%%%%%%%%
%%%%%%%%%%%%%%%%%%%%%%%%%%%%%%%%%%%%%%%%%%%%%%%%%%%%%%%%%%%%%%%%%%%%%%%%%%%
%%%%%%%%%%%%%%%%%%%%%%%%%%%%%%%%%%%%%%%%%%%%%%%%%%%%%%%%%%%%%%%%%%%%%%%%%%%




%%%%%%%%%%%%%%%%%%%%%%%%%%%%%%%%%%%%%%%%%%%%%%%%%%%%%%%%%%%%%%%%%%%%%%%%%%%
%%%%%%%%%%%%%%%%%%%%%%%%%%%%%%%%%%%%%%%%%%%%%%%%%%%%%%%%%%%%%%%%%%%%%%%%%%%
%%%%%%%%%%%%%%%%%%%%%%%%%%%%%%%%%%%%%%%%%%%%%%%%%%%%%%%%%%%%%%%%%%%%%%%%%%%
% Modified from the example in
% http://www.texample.net/tikz/examples/d-flip-flops-and-shift-register/
\makeatletter

\pgfdeclareshape{mimotresesquerda}{
  \savedmacro\antennaoffset{\def\antennaoffset{-\pgfkeysvalueof{/pgf/antenna offset}}}
  \savedmacro\espelho{\def\espelho{\pgfkeysvalueof{/pgf/espelho}}}

  % The 'minimum width' and 'minimum height' keys, not the content, determine
  % the size

  % Inherit saved anchor from rectangle (\northeast and \southwest)
  \inheritsavedanchors[from=rectangle]

  % Inherit from rectangle
  \inheritanchorborder[from=mimotresdireita]

  % Inherit normal anchors from rectangle
  \inheritanchor[from=mimotresdireita]{center}
  \inheritanchor[from=mimotresdireita]{north}
  \inheritanchor[from=mimotresdireita]{east}
  \inheritanchor[from=mimotresdireita]{south}
  \inheritanchor[from=mimotresdireita]{west}
  \inheritanchor[from=mimotresdireita]{north east}
  \inheritanchor[from=mimotresdireita]{north west}
  \inheritanchor[from=mimotresdireita]{south west}
  \inheritanchor[from=mimotresdireita]{south east}
  \inheritanchor[from=mimotresdireita]{text}
  \inheritanchor[from=mimotresdireita]{base}
  \inheritanchor[from=mimotresdireita]{mid}
  % Posso herdar outras como "mid west", "base west", etc. Veja no código do rectangle


  % Define anchors for the start of the antennas
  \anchor{A}{
    \pgf@process{\southwest}
    \pgf@xa=\pgf@x
    \pgf@process{\northeast}
    \pgf@x=\pgf@xa
    \pgf@y=.5\pgf@y
  }

  \anchor{C}{
    \pgf@process{\southwest}
    \pgf@xa=\pgf@x
    \pgf@ya=.9\pgf@y % Na coordenada southwest o valor de y é negativo
    \pgf@process{\northeast}
    \pgf@x=\pgf@xa
    \pgf@y=\pgf@ya
  }

  % B is in the middle of A and C
  \anchor{B}{
    \csname pgf@anchor@mimotresesquerda@C\endcsname
    \pgf@ya=.5\pgf@y
    \csname pgf@anchor@mimotresesquerda@A\endcsname
    \pgf@y=.5\pgf@y
    \advance\pgf@y by \pgf@ya
  }

  % Define some more useful anchors
  \anchor{first antenna base start}{\pgf@anchor@mimotresesquerda@A}
  \anchor{second antenna base start}{\pgf@anchor@mimotresesquerda@B}
  \anchor{third antenna base start}{\pgf@anchor@mimotresesquerda@C}

  \anchor{first antenna base end}{
    % Preciso usar o \csname ... \endcsname porque a âncora "first antenna
    % base start" possui espaços. Note que não pode ter espaço entre o
    % final de "first antenna base start" e o \endcsname
    \csname pgf@anchor@mimotresesquerda@first antenna base start\endcsname
    \advance\pgf@x by \antennaoffset cm%
    \advance\pgf@y by 0.2cm%
  }
  \anchor{second antenna base end}{
    \csname pgf@anchor@mimotresesquerda@second antenna base start\endcsname
    \advance\pgf@x by \antennaoffset cm%
    \advance\pgf@y by 0.2cm%
  }
  \anchor{third antenna base end}{
    \csname pgf@anchor@mimotresesquerda@third antenna base start\endcsname
    \advance\pgf@x by \antennaoffset cm%
    \advance\pgf@y by 0.2cm%
  }
  \anchor{first antenna}{
    \csname pgf@anchor@mimotresesquerda@first antenna base start\endcsname
    \advance\pgf@x by \antennaoffset cm%
    \advance\pgf@y by 0.4cm%
  }
  \anchor{second antenna}{
    \csname pgf@anchor@mimotresesquerda@second antenna base start\endcsname
    \advance\pgf@x by \antennaoffset cm%
    \advance\pgf@y by 0.4cm%
  }
  \anchor{third antenna}{
    \csname pgf@anchor@mimotresesquerda@third antenna base start\endcsname
    \advance\pgf@x by \antennaoffset cm%
    \advance\pgf@y by 0.4cm%
  }


  % Draw the rectangle box and the antennas
  \inheritbackgroundpath[from=mimotresdireita]
}

% Key to add font macros to the current font
% \tikzset{add font/.code={\expandafter\def\expandafter\tikz@textfont\expandafter{\tikz@textfont#1}}} 

% Define default style for this node
%\tikzset{flip flop/port labels/.style={font=\sffamily\scriptsize}}
\tikzset{every mimotresesquerda node/.style={draw,minimum width=1.5cm,minimum 
height=2cm,thick,inner sep=1mm,outer sep=0pt,cap=round,add 
font=\sffamily}}

\makeatother
%%%%%%%%%%%%%%%%%%%%%%%%%%%%%%%%%%%%%%%%%%%%%%%%%%%%%%%%%%%%%%%%%%%%%%%%%%%
%%%%%%%%%%%%%%%%%%%%%%%%%%%%%%%%%%%%%%%%%%%%%%%%%%%%%%%%%%%%%%%%%%%%%%%%%%%
%%%%%%%%%%%%%%%%%%%%%%%%%%%%%%%%%%%%%%%%%%%%%%%%%%%%%%%%%%%%%%%%%%%%%%%%%%%

%%% Local Variables: 
%%% mode: latex
%%% TeX-master: "figs/testnewshape.tex"
%%% TeX-PDF-mode: t
%%% End: 
