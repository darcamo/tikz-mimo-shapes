\makeatletter


% The keys that the user can provide are all the keys below, except
% "leftantennas" (use the "right antennas" or "left antennas" keys instead)
\pgfkeys{
  /pgf/.cd,
  button fill color/.initial=white
}



% xxxxxxxxxxxxxxxxxxxxxxxxxxxxxxxxxxxxxxxxxxxxxxxxxxxxxxxxxxxxxxxxxxxxxxxxx
% xxxxxxxxxxxxxxx UserTerminal Shape xxxxxxxxxxxxxxxxxxxxxxxxxxxxxxxxxxxxxx
% xxxxxxxxxxxxxxxxxxxxxxxxxxxxxxxxxxxxxxxxxxxxxxxxxxxxxxxxxxxxxxxxxxxxxxxxx
\pgfdeclareshape{UserTerminal}{
  
  % xxxxxxxxxxxxxxxxxxxxxxxxxxxxxxxxxxxxxxxxxxxxxxxxxxxxxxxxxxxxxxxxxxxxxxx
  % Macros and savedanchors common to UserTerminal and BaseStation Shapes
  \savedanchor{\center}{
    \pgf@x=0pt
    \pgf@y=0pt
  }


  % Note that pgf provides the text macros \pgfshapeminwidth and
  % pgfshapeminheight that store the minim width and minimum height,
  % respectivelly. Note that They are "TEXT macros" and not
  % "lengths". Calling them simple
  % replaces for the text.
  \savedanchor{\northeast}{
    \pgfmathsetlength\pgf@xa{\pgfshapeminwidth}
    \pgfmathsetlength\pgf@ya{\pgfshapeminheight}
    \pgf@x=0.5\pgf@xa
    \pgf@y=0.5\pgf@ya
    % NOTE: We can't use 0.5\pgfshapeminwidth directly because the minimum
    % width is stored in a "plain text macro". If minimum width is 2cm, for
    % instance, and we write "\pgf@x=.5\pgfshapeminwidth" then \pgf@x would
    % be set to 0.52cm, which is not what we want. See the explanation for
    % \savedanchor in the pgfmanual (page 627).
  }

  \savedanchor{\southwest}{
    \pgfmathsetlength\pgf@xa{\pgfshapeminwidth}
    \pgfmathsetlength\pgf@ya{\pgfshapeminheight}
    \pgf@x=-0.5\pgf@xa
    \pgf@y=-0.5\pgf@ya
  }
  % xxxxxxxxxxxxxxxxxxxxxxxxxxxxxxxxxxxxxxxxxxxxxxxxxxxxxxxxxxxxxxxxxxxxxxx

  % xxxxxxxxxxxxxxxxxxxxxxxxxxxxxxxxxxxxxxxxxxxxxxxxxxxxxxxxxxxxxxxxxxxxxxx
  % Savedanchors specific to the UserTerminal shape
  \savedanchor{\screennortheast}{
    \pgfmathsetlength\pgf@xa{\pgfshapeminwidth}
    \pgfmathsetlength\pgf@ya{\pgfshapeminheight}
    \pgf@x=0.45\pgf@xa
    \pgf@y=0.45\pgf@ya
  }

  \savedanchor{\screensouthwest}{
    \pgfmathsetlength\pgf@xa{\pgfshapeminwidth}
    \pgfmathsetlength\pgf@ya{\pgfshapeminheight}
    \pgf@x=-0.45\pgf@xa
    \pgf@y=-0.35\pgf@ya
  }

  \savedanchor{\buttonsouthwest}{
    \pgfmathsetlength\pgf@xa{\pgfshapeminwidth}
    \pgfmathsetlength\pgf@ya{\pgfshapeminheight}
    \pgf@x=0pt
    \pgf@y=-0.425\pgf@ya
    \advance\pgf@x by -0.15\pgf@xa
    \advance\pgf@y by -0.04\pgf@ya
  }

  \savedanchor{\buttonnortheast}{
    \pgfmathsetlength\pgf@xa{\pgfshapeminwidth}
    \pgfmathsetlength\pgf@ya{\pgfshapeminheight}
    \pgf@x=0pt
    \pgf@y=-0.425\pgf@ya
    \advance\pgf@x by 0.15\pgf@xa
    \advance\pgf@y by 0.04\pgf@ya
  }

  % xxxxxxxxxxxxxxxxxxxxxxxxxxxxxxxxxxxxxxxxxxxxxxxxxxxxxxxxxxxxxxxxxxxxxxx

  % xxxxxxxxxx Regular Anchors xxxxxxxxxxxxxxxxxxxxxxxxxxxxxxxxxxxxxxxxxxxx
  \anchor{text}
  {
    % This will get the center of the screen
    \pgfpointlineattime{0.5}{\screensouthwest}{\screennortheast}
    % Lets put the center of the text in the center of the screen
    \advance\pgf@x by -0.5\wd\pgfnodeparttextbox
    \advance\pgf@y by -0.5\ht\pgfnodeparttextbox
  }

  \anchor{button}{
    \pgfpointlineattime{0.5}{\buttonsouthwest}{\buttonnortheast}
  }

  \anchor{screen}{
    \pgfpointlineattime{0.5}{\screensouthwest}{\screennortheast}
  }

  % % xxxxxxxxxx ERASE LATER these two anchors xxxxxxxxxxxxxxxxxxxxxxxxxxxxxx
  % \anchor{button south west}{\buttonsouthwest}
  % \anchor{button north east}{\buttonnortheast}


  % We define the savedanchors "northeast" and "southwest". With these two
  % defined we can inherit code for the regular anchors from the
  % rectangle shape (even thought the northeast and southwest saved anchors
  % in the rectangle shape are different from ours)..
  \inheritanchor[from=rectangle]{center}
  \inheritanchor[from=rectangle]{north}
  \inheritanchor[from=rectangle]{east}
  \inheritanchor[from=rectangle]{south}
  \inheritanchor[from=rectangle]{west}
  \inheritanchor[from=rectangle]{north east}
  \inheritanchor[from=rectangle]{north west}
  \inheritanchor[from=rectangle]{south west}
  \inheritanchor[from=rectangle]{south east}
  %\inheritanchor[from=rectangle]{text}
  % xxxxxxxxxxxxxxxxxxxxxxxxxxxxxxxxxxxxxxxxxxxxxxxxxxxxxxxxxxxxxxxxxxxxxxx

  % We can even inherit the "angle" anchors.
  \inheritanchorborder[from=rectangle]%

  \backgroundpath{
    % Scope for Terminal Body
    \begin{pgfscope}
      \pgfsetcornersarced{\pgfpoint{0.1\pgfshapeminheight}{0.1\pgfshapeminheight}}
      \pgfpathrectanglecorners
      {\southwest}
      {\northeast}
      \pgfusepath{stroke,fill}
    \end{pgfscope}
    
    % Scope for the terminal Screen
    \begin{pgfscope}
      \pgfsetcornersarced{\pgfpoint{0.1\pgfshapeminheight}{0.1\pgfshapeminheight}}
      \pgfsetfillcolor{white}
      \pgfpathrectanglecorners
      {\screensouthwest}
      {\screennortheast}
      \pgfusepath{stroke,fill}
    \end{pgfscope}
    
    % Scope for the button
    \begin{pgfscope}
      \pgfsetcornersarced{\pgfpoint{0.03\pgfshapeminheight}{0.03\pgfshapeminheight}}
      \pgfsetfillcolor{\pgfkeysvalueof{/pgf/button fill color}}
      %\pgfsetfillcolor{white}
      \pgfpathrectanglecorners
      {\buttonsouthwest}
      {\buttonnortheast}
      \pgfusepath{stroke,fill}
    \end{pgfscope}
  }
}
% xxxxxxxxxxxxxxxxxxxxxxxxxxxxxxxxxxxxxxxxxxxxxxxxxxxxxxxxxxxxxxxxxxxxxxxxx
% xxxxxxxxxxxxxxx END xxxxxxxxxxxxxxxxxxxxxxxxxxxxxxxxxxxxxxxxxxxxxxxxxxxxx
% xxxxxxxxxxxxxxxxxxxxxxxxxxxxxxxxxxxxxxxxxxxxxxxxxxxxxxxxxxxxxxxxxxxxxxxxx


% xxxxxxxxxxxxxxxxxxxxxxxxxxxxxxxxxxxxxxxxxxxxxxxxxxxxxxxxxxxxxxxxxxxxxxxxx
% xxxxxxxxxxxxxxx BaseStation Shape xxxxxxxxxxxxxxxxxxxxxxxxxxxxxxxxxxxxxxx
% xxxxxxxxxxxxxxxxxxxxxxxxxxxxxxxxxxxxxxxxxxxxxxxxxxxxxxxxxxxxxxxxxxxxxxxxx
\pgfdeclareshape{BaseStation}{
  % Inherit all savedanchors from UserTerminal. We are interested in
  % northeast and southwest. Note that this will also inherits the saved
  % macros.
  \inheritsavedanchors[from=UserTerminal]


  % xxxxx Define savedanchors specific to the BaseStation shape xxxxxxxxxxx
  \savedanchor{\BaseAsouthwest}{
    \pgfmathsetlength\pgf@xa{\pgfshapeminwidth}
    \pgfmathsetlength\pgf@ya{\pgfshapeminheight}
    \pgf@y=-0.5\pgf@ya
    \pgf@x=-0.2\pgf@xa
  }

  \savedanchor{\BaseAnortheast}{
    \pgfmathsetlength\pgf@xa{\pgfshapeminwidth}
    \pgfmathsetlength\pgf@ya{\pgfshapeminheight}
    \pgf@y=-0.05\pgf@ya
    \pgf@x=0.2\pgf@xa
  }

  \savedanchor{\BaseBsoutheast}{
    \pgfmathsetlength\pgf@xa{\pgfshapeminwidth}
    \pgfmathsetlength\pgf@ya{\pgfshapeminheight}
    \pgf@y=-0.05\pgf@ya
    \pgf@x=0.15\pgf@xa
  }

  \savedanchor{\BaseBnorthwest}{
    \pgfmathsetlength\pgf@xa{\pgfshapeminwidth}
    \pgfmathsetlength\pgf@ya{\pgfshapeminheight}
    \pgf@y=0.15\pgf@ya
    \pgf@x=-0.15\pgf@xa
  }

  \savedanchor{\BaseCsouthwest}{
    \pgfmathsetlength\pgf@xa{\pgfshapeminwidth}
    \pgfmathsetlength\pgf@ya{\pgfshapeminheight}
    \pgf@y=0.15\pgf@ya
    \pgf@x=-0.1\pgf@xa
  }
  
  \savedanchor{\BaseCnortheast}{
    \pgfmathsetlength\pgf@xa{\pgfshapeminwidth}
    \pgfmathsetlength\pgf@ya{\pgfshapeminheight}
    \pgf@y=0.30\pgf@ya
    \pgf@x=0.1\pgf@xa
  }
  
  % Maybe I don't need this savedanchor
  \savedanchor{\MiddleAntennasouth}{
    \pgfmathsetlength\pgf@ya{\pgfshapeminheight}
    \pgf@x=0pt
    \pgf@y=0.29\pgf@ya
  }

  \savedanchor{\AntennaConnectorsouthwest}{
    \pgfmathsetlength\pgf@xa{\pgfshapeminwidth}
    \pgfmathsetlength\pgf@ya{\pgfshapeminheight}
    \pgf@x=-0.5\pgf@xa
    \pgf@y=0.38\pgf@ya
  }

  \savedanchor{\AntennaConnectornortheast}{
    \pgfmathsetlength\pgf@xa{\pgfshapeminwidth}
    \pgfmathsetlength\pgf@ya{\pgfshapeminheight}
    \pgf@x=0.5\pgf@xa
    \pgf@y=0.42\pgf@ya
  }

  \savedanchor{\MiddleAntennasouthwest}{
    \pgfmathsetlength\pgf@xa{\pgfshapeminwidth}
    \pgfmathsetlength\pgf@ya{\pgfshapeminheight}
    \pgf@x=-0.15\pgf@xa
    \pgf@y=0.29\pgf@ya
  }

  \savedanchor{\MiddleAntennanortheast}{
    \pgfmathsetlength\pgf@xa{\pgfshapeminwidth}
    \pgfmathsetlength\pgf@ya{\pgfshapeminheight}
    \pgf@x=0.15\pgf@xa
    \pgf@y=0.5\pgf@ya
  }

  \savedanchor{\LeftAntennasouthwest}{
    \pgfmathsetlength\pgf@xa{\pgfshapeminwidth}
    \pgfmathsetlength\pgf@ya{\pgfshapeminheight}
    \pgf@x=-0.5\pgf@xa
    \pgf@y=0.29\pgf@ya
  }

  \savedanchor{\LeftAntennanortheast}{
    \pgfmathsetlength\pgf@xa{\pgfshapeminwidth}
    \pgfmathsetlength\pgf@ya{\pgfshapeminheight}
    \pgf@x=-0.25\pgf@xa
    \pgf@y=0.5\pgf@ya
  }

  \savedanchor{\RightAntennasouthwest}{
    \pgfmathsetlength\pgf@xa{\pgfshapeminwidth}
    \pgfmathsetlength\pgf@ya{\pgfshapeminheight}
    \pgf@x=0.25\pgf@xa
    \pgf@y=0.29\pgf@ya
  }

  \savedanchor{\RightAntennanortheast}{
    \pgfmathsetlength\pgf@xa{\pgfshapeminwidth}
    \pgfmathsetlength\pgf@ya{\pgfshapeminheight}
    \pgf@x=0.5\pgf@xa
    \pgf@y=0.5\pgf@ya
  }
  
  % xxxxxxxxxx Regular Anchors xxxxxxxxxxxxxxxxxxxxxxxxxxxxxxxxxxxxxxxxxxxx
  % We define the savedanchors "northeast" and "southwest". With these two
  % defined we can inherit code for the regular anchors from the
  % UserTerminal shape (even thought the northeast and southwest saved anchors
  % in the UserTerminal shape are different from ours)..
  \inheritanchor[from=UserTerminal]{center}
  \inheritanchor[from=UserTerminal]{north}
  \inheritanchor[from=UserTerminal]{east}
  \inheritanchor[from=UserTerminal]{south}
  \inheritanchor[from=UserTerminal]{west}
  \inheritanchor[from=UserTerminal]{north east}
  \inheritanchor[from=UserTerminal]{north west}
  \inheritanchor[from=UserTerminal]{south west}
  \inheritanchor[from=UserTerminal]{south east}
  \inheritanchor[from=rectangle]{text}
  % xxxxxxxxxxxxxxxxxxxxxxxxxxxxxxxxxxxxxxxxxxxxxxxxxxxxxxxxxxxxxxxxxxxxxxx

  % We can even inherit the "angle" anchors.
  \inheritanchorborder[from=UserTerminal]%
  
  \backgroundpath{
    % Scope for the Base
    \begin{pgfscope}
      % Draw the bottom base
      \pgfpathrectanglecorners
      {\BaseAsouthwest}
      {\BaseAnortheast}

      % Draw the middle base
      \pgfpathrectanglecorners
      {\BaseBsoutheast}
      {\BaseBnorthwest}

      % Draw the bottom base
      \pgfpathrectanglecorners
      {\BaseCsouthwest}
      {\BaseCnortheast}

      \pgfusepath{stroke,fill}
    \end{pgfscope}

    % Scope for the antenna connector
    \begin{pgfscope}
      % Draw Antenna connector
      \pgfpathrectanglecorners
      {\AntennaConnectorsouthwest}
      {\AntennaConnectornortheast}

      \pgfusepath{stroke,fill}
    \end{pgfscope}

    % Scope for the antennas
    \begin{pgfscope}
      % Draw the middle antenna
      \pgfpathrectanglecorners
      {\MiddleAntennasouthwest}
      {\MiddleAntennanortheast}
      
      % Draw the Left antenna
      \pgfpathrectanglecorners
      {\LeftAntennasouthwest}
      {\LeftAntennanortheast}
      
      % Draw the Right antenna
      \pgfpathrectanglecorners
      {\RightAntennasouthwest}
      {\RightAntennanortheast}

      \pgfusepath{stroke,fill}
    \end{pgfscope}
  }
}
% xxxxxxxxxxxxxxxxxxxxxxxxxxxxxxxxxxxxxxxxxxxxxxxxxxxxxxxxxxxxxxxxxxxxxxxxx
% xxxxxxxxxxxxxxx END xxxxxxxxxxxxxxxxxxxxxxxxxxxxxxxxxxxxxxxxxxxxxxxxxxxxx
% xxxxxxxxxxxxxxxxxxxxxxxxxxxxxxxxxxxxxxxxxxxxxxxxxxxxxxxxxxxxxxxxxxxxxxxxx


\makeatother

%%% Local Variables: 
%%% mode: latex
%%% TeX-master: "testbasestation.tex"
%%% TeX-PDF-mode: t
%%% End: 
